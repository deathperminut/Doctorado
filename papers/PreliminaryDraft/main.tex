\documentclass[11pt,a4paper]{article}

% ========================
% PAQUETES BÁSICOS
% ========================
\usepackage[utf8]{inputenc}
\usepackage[spanish,es-tabla]{babel}
\usepackage{amsmath,amssymb}
\usepackage{graphicx}
\usepackage{hyperref}
\usepackage{geometry}
\usepackage{float}
\usepackage{booktabs}

% Configuración
\geometry{margin=1in}
\hypersetup{
    colorlinks=true,
    linkcolor=blue,
    urlcolor=cyan,
    citecolor=blue
}

% ========================
% INFORMACIÓN
% ========================
\title{Borrador Preliminar\\
\large Análisis de Dominios Magnéticos con CNNs}

\author{
    Juan Sebastián Méndez Rondón
}

\date{\today}

% ========================
% DOCUMENTO
% ========================
\begin{document}

\maketitle

\begin{abstract}
Este borrador preliminar presenta los avances iniciales en la aplicación de redes neuronales convolucionales para el análisis de dominios magnéticos. Se exploran diferentes arquitecturas y metodologías para la predicción de parámetros físicos.
\end{abstract}

\section{Introducción}

Este documento sirve como borrador preliminar para organizar ideas y resultados iniciales del proyecto de doctorado.

\subsection{Objetivos}

\begin{itemize}
    \item Explorar arquitecturas CNN para dominios magnéticos
    \item Validar metodología de preprocesamiento
    \item Establecer baseline de performance
\end{itemize}

\section{Metodología Preliminar}

\subsection{Datos}

Trabajamos con configuraciones de spin de simulaciones Monte Carlo:
\begin{itemize}
    \item Database Jex2T: 54,044 configuraciones
    \item Database KDMT: En construcción
\end{itemize}

\subsection{Modelos Explorados}

\begin{table}[H]
\centering
\caption{Modelos evaluados hasta ahora}
\begin{tabular}{lc}
\toprule
\textbf{Arquitectura} & \textbf{Estado} \\
\midrule
DenseNet121 & En progreso \\
ResNet50 & Planeado \\
EfficientNet & Planeado \\
\bottomrule
\end{tabular}
\end{table}

\section{Resultados Preliminares}

\textit{[Sección en construcción]}

% Espacio para figuras
\begin{figure}[H]
\centering
% \includegraphics[width=0.6\textwidth]{figures/placeholder.pdf}
\caption{Placeholder para resultados iniciales}
\end{figure}

\section{Próximos Pasos}

\begin{enumerate}
    \item Completar entrenamiento de DenseNet
    \item Implementar análisis UMAP
    \item Generar visualizaciones Grad-CAM
    \item Escribir sección de resultados completa
\end{enumerate}

\section{Notas y Observaciones}

\textit{[Espacio para notas durante el desarrollo]}

\subsection{Ideas}

\begin{itemize}
    \item Probar data augmentation
    \item Explorar diferentes tamaños de imagen
    \item Considerar ensemble methods
\end{itemize}

\subsection{Dudas}

\begin{itemize}
    \item ¿Mejor métrica para evaluar: R², MAPE, o ambas?
    \item ¿Cómo manejar el desbalance en temperatura?
\end{itemize}

\section{Referencias Temporales}

\textit{[Añadir referencias según avance el trabajo]}

\begin{itemize}
    \item Huang et al. - DenseNet paper
    \item Selvaraju et al. - Grad-CAM paper
    \item McInnes et al. - UMAP paper
\end{itemize}

% Bibliografía (opcional para borrador)
% \bibliographystyle{plain}
% \bibliography{references}

\end{document}
